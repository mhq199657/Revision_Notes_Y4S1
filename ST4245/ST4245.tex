\PassOptionsToPackage{svgnames}{xcolor}
\documentclass[12pt]{article}



\usepackage[margin=1in]{geometry}  
\usepackage{graphicx}             
\usepackage{amsmath}              
\usepackage{amsfonts}              
\usepackage{framed}               
\usepackage{amssymb} 
\usepackage{array}
\usepackage{amsthm}
\usepackage[nottoc]{tocbibind}
\usepackage{bm}
\usepackage{enumitem}


  \newcommand\norm[1]{\left\lVert#1\right\rVert}
\setlength{\parindent}{0cm}
\setlength{\parskip}{0em}
\newcommand{\Lim}[1]{\raisebox{0.5ex}{\scalebox{0.8}{$\displaystyle \lim_{#1}\;$}}}
\newtheorem{definition}{Definition}[section]
\newtheorem{theorem}{Theorem}[section]
\newtheorem{notation}{Notation}[section]
\theoremstyle{definition}
\DeclareMathOperator{\arcsec}{arcsec}
\DeclareMathOperator{\arccot}{arccot}
\DeclareMathOperator{\arccsc}{arccsc}
\DeclareMathOperator{\PV}{PV}
\DeclareMathOperator{\TV}{TV}
\DeclareMathOperator{\diff}{d}
\DeclareMathOperator{\expec}{E}
\DeclareMathOperator{\var}{Var}
\DeclareMathOperator{\cov}{Cov}
\DeclareMathOperator{\Corr}{Corr}
\DeclareMathOperator{\CE}{CE}
\DeclareMathOperator{\RP}{RP}
\DeclareMathOperator{\LogN}{logN}
\DeclareMathOperator{\VaR}{VaR}
\DeclareMathOperator{\ES}{ES}
\DeclareMathOperator{\SK}{SK}
\DeclareMathOperator{\Kur}{Kur}
\DeclareMathOperator{\ExKur}{Ex.Kur}
\DeclareMathOperator{\ARMA}{ARMA}
\DeclareMathOperator{\AR}{AR}
\DeclareMathOperator{\ARCH}{ARCH}
\DeclareMathOperator{\GARCH}{GARCH}
\DeclareMathOperator{\ARIMA}{ARIMA}
\newcommand\cf[1]{\mathbf{#1}}
\setcounter{tocdepth}{1}
\setcounter{section}{-1}
\begin{document}

\title{Revision notes - ST4245}
\author{Ma Hongqiang}
\maketitle
\tableofcontents

\clearpage
%\twocolumn
\section{Introduction}
\subsection{Return}
We model the movement of the price $P_t$ as a random walk by assuming the profit at that period $v_t:= P_t -P_{t-1}, t= 1,\ldots$ are IID random variable with 
\[
E(v_1) = \mu, \cov(v_1)= \sigma^2
\] 
\begin{definition}[Random Walk]
\hfill\\\normalfont Let $P_0$ be a point, and $P_t= P_0 + v_1 + \cdots + v_t$. Process $\{P_0, P_1, \ldots\}$ is called a \textbf{random walk} and $\{v_1, v_2,\ldots\}$ are its step sizes.\\
We have 
\[
E(P_t\mid P_0) = P_0 + t\mu,\;\;\;\cov(P_t\mid P_0)= \sigma^2 t
\]
Here, $\mu$ is called the drift. $\sigma$ is called the volatility.\\
If $v_i$ are normally distributed, we can the process a \textit{normal} random walk.\\ 
\end{definition}
\begin{definition}[Gross Return]
\hfill\\\normalfont The gross return over $k$ periods is $G_t(k) = \frac{P_t}{P_{t-k}}$.
\end{definition}
\begin{definition}[Net Return]
\hfill\\\normalfont Assuming no dividend, the net return over $k$ holding periods is
\[
R_t(k) = \frac{P_t}{P_{t-k}} - 1
\]
\end{definition}
\begin{definition}[Log Return]
\hfill\\\normalfont The log return over $k$ periods is
\[
r_t(k)=\log(\frac{P_t}{P_{t-k}})=\log(1+R_t(k))
\]
\end{definition}
\textbf{Remark}:
\begin{itemize}
  \item when $|R_t(k)|$ is small, $r_t\approx R_t$.
  \item $k$-period log return is the sum of single-period log returns.
\end{itemize}
\begin{definition}[Adjustment for Dividend]
\hfill\\\normalfont If a dividend(interest) $D_t$ is paid prior to time $t$, so the initial price of day $t$ is 
\[
P_{t-1}-D_t
\]
Then the gross return on day $t$ is
\[
1+R_t=\frac{P_t}{P_{t-1}-D_t}
\]
Therefore, to maintain the returns' nice-looking equation, we need to let $\frac{P_t}{P_{t-1}-D_t}:=\frac{P_t}{P'_{t-1}}:=\frac{P_t}{P_{t-1}\times a}$.
Apparently, $a=1-\frac{D_t}{P_{t-1}}$.
\end{definition}
In the analysis of financial data over a long period of time, we should use adjusted price.
\begin{definition}[Excess Return]
\hfill\\\normalfont Excess return is the difference $r_t-r_t^\ast$ betwwen the asset's log return $r_t$ and the log return of $r_t^\ast$ on some reference asset(usually risk free).
\end{definition}
\begin{definition}[Returns of a Portfolio]
\hfill\\\normalfont Suppose onehas a portfolio consisting of $p$ different assets. Let $w_i$ be weights such that $\sum_{i=1}^p w_i = 1$. Then, the value of the asset $i$ is $w_iP_t$ when the total value of the portfolio is $P_t$.\\
Suppose $R_{it}$ and $r_{it}$ are the net return and log return of the asset $i$ at time $t$. The value of portfolio provided by asset $i$ at time $t$ is $w_iP_{t-1}(1+R_{it})$. So the total value of a portfolio is
\[
P_t=\sum_{i=1}^p w_{i}P_{t-1}(1+R_{it}) = (1+\sum_{i=1}^p w_iR_{it})P_{t-1}
\]
Overall net return $R_t$ and the log return $r_t$ of the portfolio are respectively
\[
R_t=\frac{P_t}{P_{t-1}}=\sum_{i=1}^p w_iR_{it}
\]
and
\[
r_t=\log(1+\sum_{i=1}^p w_iR_{it})\approx \sum_{i=1}^p w_iR_{it} \approx\sum_{i=1}^p w_ir_{it}
\]
\end{definition}
\begin{theorem}[Random Walk Model for Log Return]
\hfill\\\normalfont The random walk hypothesis states that the single-period log returns, $r_t=\log(P_t)-\log(P_{t-1})$ are independent. Thus, $\log(P_t) - log(P_0)$ is a random walk if $r_t$ are IID.\\
Sometimes, we further assume
\begin{itemize}
  \item $r_t\sim N(\mu, \sigma^2)$
  \item This leads to $\log(P_t)-\log(P_{t-k})\sim N(k\mu,k\sigma^2)$.
  \item And also, $\frac{P_t}{P_{t-k}}$ will be lognormal.
\end{itemize}
\end{theorem}
\subsection{Geometric Random Walks}
We have $P_t=P_0\exp(r_t+\cdots +r_1)$. We call such aprocess whose \textit{logarithm} a random walk a \textbf{geometric random walk}. If $r_1,\ldots$ are IID $N(\mu,\sigma^2)$, then $P_t$ is a log-normal geometric random walk with parameter $(\mu, \sigma^2)$.
\subsection{Interest Rate}
In the case of a risk-free asset, the rate of return is called interest rate. Ifthe interest rate isconstant $R$ compounded once per unit period, then the value of the risk-free asset at time $t$ is $P_t=P_0(1+R)^t$.

\clearpage
\section{Exploratory Data Analysis, Moments}
\subsection{Useful Distribution}
\begin{itemize}
  \item Normal Distribution: $N(\mu, \sigma^2)$
  \[
f(x;\mu,\sigma^2) = \frac{1}{\sigma\sqrt{2\pi}}\exp(-\frac{1}{2}(\frac{x-\mu}{\sigma})^2)
  \]
  \item $t$-distribution: $t(\nu)$
  \[
f(x)=\frac{\Gamma(\frac{\nu + 1}{2})}{\sqrt{\nu\pi}\Gamma(\frac{\nu}{2})}(1+\frac{x^2}{\nu})^{-\frac{\nu+1}{2}}
  \]
  \item Log-normal distribution:$\LogN(\mu,\sigma)$
  \[
f(x;\mu,\sigma) =\frac{1}{x\sigma\sqrt{2\pi}}\exp(-\frac{(\log(x)-\mu)^2}{2\sigma^2}),\;\;\;x>0
  \]
\end{itemize}
\subsection{Moments}
\begin{definition}[Moment, Central Moment]
\hfill\\\normalfont
Let $X$ be a random variable. The $k$th \textbf{moment} of $X$ is $E(X^k)$. The $k$th \textbf{central moment} is defined as
\[
\mu_1=E(X), \;\mu_k=E[(X-E(X))^k]
\]
\end{definition}
\textbf{Remark}:
\begin{itemize}
\item First moment is the mean $\mu$.
\item Second central moment is variance $\sigma^2$.
\item If $X$ is the return, the we call $\sigma(X)$ the risk.
\end{itemize}
\begin{theorem}[Useful Results of Moments]
\hfill\\\normalfont
\begin{itemize}
  \item If $X\sim N(\mu,\sigma^2)$,then
  \[
E(X)=\mu, \cov(X)=\sigma^2, \mu_3=0, \mu_4=3\sigma^4
  \]
  \item If $X\sim t(\nu)$, then
  \[
E(X)=0, \cov(X)=\frac{\nu}{\nu-2}(\text{ for }\nu>2), \mu_3=0, \mu_4=\frac{3}{\nu-4}\frac{\nu^2}{\nu-2}(\text{ for }\nu>4)
  \]
  \item If $X\sim\LogN(\mu, \sigma^2)$, then
  \[
E(X^s)=\exp(s\mu +\frac{1}{2}s^2\sigma^2)
  \]
\end{itemize}
\begin{definition}[Sharpe Ratio]
\hfill\\\normalfont For random variable $X$, the Sharpe ratio is defined as
\[
SR=\frac{E(X)}{\sigma(X)}
\]
In finance, the Sharpe Ratio of return $R$ is
\[
SR(R)=\frac{E(R-r_f)}{\sigma(R)}
\]
\end{definition}
\begin{definition}[Sample Mean and Variance, and Estimation of Population Mean and Variance]
With sample $Y_1, \ldots, Y_n$,
\[
\hat{\mu}=\bar{Y}=\frac{1}{n}\sum_{i=1}^n Y_i\;\;\;\;\hat{\sigma}^2 = \frac{1}{n}\sum_{i=1}^n(Y_i-\bar{Y})^2
\]
If $Y_1,\ldots, Y_n$ are IID with mean $\mu$ and standard deviation $\sigma^2$. Then approximately
\begin{itemize}
  \item $\sqrt{n}(\hat{\mu}-\mu)\sim N(0, \sigma^2)$
  \item $\sqrt{n}(\hat{\sigma^2}-\sigma^2)\sim N(0, 2\sigma^4)$
  \item $\sqrt{n}(\hat{\sigma}-\sigma)\sim N(0, \frac{1}{2}\sigma^2)$
\end{itemize}
\end{definition}
We can estimate the Sharpe Ratio by $\hat{SR}=\frac{\hat{\mu}}{\hat{\sigma}}$, with
\[
\hat{SR}-SR\sim N(0,\frac{1}{n}(1+\frac{1}{2}SR^2))
\]
\subsection{Quantile, Value At Risk, Expected Shortfall}
\begin{definition}[Quantile]
Suppose $X\sim F(x)$, for any $0<q<1$, the $q$th quantile of $X$ is defined as
\[
Q_q(X)=\max\{x:P(X<x)\leq q\}
\]
\end{definition}
\begin{definition}[Value At Risk]
\hfill\\\normalfont If $X$ is the investment , the $-Q_q(X)$ is called the $100q\%$ \textbf{value at risk}(VaR):
\[
\VaR_q(X)=-Q_q(X)=-\max\{v:F(v)\leq q\}
\]
\end{definition}
It is easy to see that
\begin{itemize}
  \item $\VaR_q(X+c)=\VaR_q(X)-c$
  \item If $X\leq Y$, the $\VaR_q(X)\geq \VaR_q(Y)$
  \item $\VaR_q(\lambda X)=\lambda \VaR_q(X)$ for any constant $\lambda > 0$
\end{itemize}
\begin{definition}[Expected Shortfall]
\hfill\\\normalfont The expected shortfall at level $q$ is
\[
\ES_q(X)=\frac{1}{q}\int_0^q\VaR_\alpha(X)\diff \alpha
\]
\end{definition}
It is easy to see that
\begin{itemize}
  \item $\ES_q(X+c)=\ES_q(X)-c$
  \item $X\leq Y\Rightarrow ES_q(X)\geq ES_q(Y)$
  \item $ES_q(\lambda X)=\lambda ES_q(X)$ for $\lambda>0$.
\end{itemize}
An alternative formula for expected shortfall is
\[
ES_q(X)=-E(X\mid X<-\VaR_q(X))=-\frac{1}{q}\int_{-\infty}^{-\VaR_q(X)}xf(x)\diff x
\]
\begin{definition}[Conherent Risk Measures]
\hfill\\\normalfont A risk measure which satisfies the following four properties is coherent:
\begin{itemize}
  \item Drift Invariance: $\rho(r+c)=\rho(r)-c$
  \item Homogeneity: $\rho(\lambda r)=\lambda\rho(r)$ for any $\lambda>0$.
  \item Monotonicity: for a pair $(r_1, r_2)$, if $r_1\geq r_2$, then $\rho(r_1)\leq \rho(r_2)$.
  \item Subadditivity: $\rho(r_1+r_2)\leq \rho(r_1)+\rho(r_2)$
\end{itemize}
\end{definition}
\begin{definition}[Skewness]
\hfill\\\normalfont Skewness coefficient of $X$ measures the degree of assymetry, and is measured as
\[
\SK(X) = \frac{\mu_3}{\sigma^3} = \frac{\mu_3}{\mu_2^2}
\]
\begin{itemize}
  \item $\SK(X)=0$ denotes symmetric distribution.
  \item $\SK(X)>0$ denotes positive skewness, which indicates that the distributionhas a long right tail compared to left tail.
  \item $\SK(X)>0$ denotes negative skewness.
\end{itemize}
\end{definition}
\begin{definition}[Kurtosis]
\hfill\\\normalfont The kurtosis of a random variable $Y$ is defined as
\[
\Kur(Y) = \frac{\mu_4}{\sigma^4} =\frac{\mu_4}{\mu_2^2}
\]
We also define Kurtosis excess:
\[
\ExKur(Y) = \Kur(Y)-3
\]
\end{definition}
Kurtosis is a measure of whether the data are heavy-tailed or light-tailed relative to a normal distribution. distribution has heavy tail if kurtosis is greater than 3, while normal has kurtosis 3.
\begin{theorem}[Kurtosis of common distributions]
\hfill\\\normalfont
\begin{itemize}
  \item $N(\mu, \sigma^2)$ has kurtosis 3.
  \item $t(\nu)$ has kurtosis $3+\frac{6}{\nu -4}$
  \item $\LogN(\mu, \sigma^2)$ has kurtosis $e^{4\sigma^2}+2e^{3\sigma^2} + 3e^{2\sigma^2}-3$.
\end{itemize}
\end{theorem}
\begin{theorem}[Estimation of Skewness and Kurtosis]
\hfill\\\normalfont Suppose $Y_1, \ldots Y_n$ are samples from a distribution. Let sample mean and standard deviation be $\bar{Y}$ and $s$. Then the sample skewness and kurtosis are
\[
\hat{SK}=\frac{1}{n}\sum_{i=1}^n (\frac{Y_i-\bar{Y}}{s})^3
\]
and
\[
\hat{Kur}=\frac{1}{n}\sum_{i=1}^n(\frac{Y_i-\bar{Y}}{s})^4
\]
and
\[
\hat{\mu}_k= \frac{1}{n}\sum_{i=1}^n(Y_i-\bar{Y})^k
\]
\end{theorem}
\begin{theorem}[Test of Skewness]
\hfill\\\normalfont Given Null hypothesis $H_0: X$ is symmetric, under $H_0$, we have
\[
\sqrt{n}\hat{SK}\sim N(0,\sigma_{SK}^2)
\]
where $\sigma_{SK}^2\approx 9 + \frac{\hat{\mu}_6}{\hat{\mu}_2^3}-6\frac{\hat{\mu_4}}{\hat{\mu}_2^2}$.
If $X$ has normal distribution, the $\sigma_{SK}^2 = 6$.
\end{theorem}
\begin{theorem}[Test of Kurtosis]
\hfill\\\normalfont Given Null hypothesis $H_0': X$ has kurtosis 3, under $H_0'$, we have
\[
\sqrt{n}(\hat{Kur}-3)\sim N(0, \sigma_{kurt}^2)
\]
where $\sigma_{kurt}^2 \approx 24 _ 6\frac{\hat{\mu}_3^2}{\hat{\mu}_2^3} - 8\frac{\hat{\mu}_3\hat{\mu}_5}{\hat{\mu}_2^4}$.
If $X$ hasnormal distribution, $\sigma_{kurt}^2= 24$.
\end{theorem}
\begin{theorem}[Test of Normality]
\hfill\\\normalfont Jarque-Bera test check whether data have the skewnessand kurtosis matching a normal distribution
\[
JB=\frac{n-k+1}{6}(\hat{SK}^2+\frac{1}{4}(\hat{Kurt}-3)^2)
\]
where $n$ is number of observations, and $k$ is number of regressionin a model, if the data is the residuals, or $k = 0$ if it is observed.\\
If data is from normal distribution, then 
\[
JB\sum\chi^2(2)
\]
\end{theorem}
\begin{definition}[Heavy Tailed Distribution]
\hfill\\\normalfont The distribution of r.v. $X$ is said to have a heavy right tail if
\[
\lim_{x\to\infty}e^{\lambda x}P(X>x)=\infty \;\;\text{for all }\lambda>0
\]
if heavy left tail if
\[
\lim_{x\to-\infty}e^{\lambda|x|}P(X<x)=\infty \;\;\text{for all }\lambda>0
\]
If
\[
\lim_{x\to\infty}\frac{P(X>x)}{P(Y>x)}\to\infty\text{ or }\lim_{x\to-\infty}\frac{P(X<x)}{P(Y<x)}\to\infty
\]
we say $X$ is heavier tail than $Y$.
\end{definition}
\begin{definition}[Fat Tailed Distribution]
\hfill\\\normalfont If $P(|X|>z)\propto z^{-\alpha}$for some $\alpha>0$.
\end{definition}

Normal distribution does not have heavy tail, but $t$ has, and log normal has right heavy tail.\\
For MLE, refer to ST2132 note.
\begin{definition}[AIC, BIC]
\hfill\\\normalfont 
Akaike's Information Criterion(AIC) is defined as
\[
AIC=-2\log\{\mathcal{L}(\hat{\theta}_{ML}\}+ 2p
\]
and Bayesian Information Criterion(BIC) is defined as
\[
BIC=-2\log\{\mathcal{L}(\hat{\theta}_{ML}\} + \log(n)p
\]
where $p$ equals the number of parameters in the model/distribution and $n$ the sample size. We call the last term the complexity penality.\\
A distribution with smaller AIC/BIC is preferred.
\end{definition}
\clearpage
\section{Multivariate Statictical Models}
\begin{theorem}[Multivariate Normal and {$t$}]
\hfill\\\normalfont Multivariate normal $N(\mathbf{\mu}, \mathbf{\Sigma})$ has pdf
\[
f_{\mathbf{x}}(x_1,\ldots, x_p)=\frac{1}{(2\pi)^{k/2}|\mathbf{\Sigma}|^{1/2}}\exp(-\frac{1}{2}(\mathbf{x}-\mathbf{\mu})^T\mathbf{\sigma}^{-1}(\mathbf{x}-\mathbf{\mu}))
\]
where $|\mathbf{\sigma}|$ is the determinant of $\mathbf{\sigma}=(\sigma_{ij})_{i\in (1, n), j\in (1, n)}$.\\
The marginal distribution of $\mathbf{x}_k$ is $N(\mu_k,\sigma_{kk}=\sigma_k^2)$.\\
Multivariate $t$ distribution $t(\nu, \mu, \sigma)$ has pdf
\[
f_{\mathbf{x}}(x_1,\ldots, x_p)=\frac{\Gamma[(\nu + p)/2]}{\Gamma(\nu/2)(\nu\pi)^{p/2}}|\mathbf{\sigma}|^{-1/2}[1+\frac{1}{\nu}(\mathbf{x}-\mathbf{\nu})^T\mathbf{\sigma}^{-1}(\mathbf{x}-\mathbf{\mu})]^{-(\nu+p)/2}
\]
\end{theorem}
\begin{theorem}[Hypothesis Testing on correlation]
\hfill\\\normalfont Given null hypothesis $H_0:\rho_{X,Y}=0$. Under $H_0$, we have
\[
\sqrt{n-3}r_{X,Y}\sim N(0,1)
\]
\end{theorem}
\begin{definition}[Portfolio]
\hfill\\\normalfont Let $R_i, i = 1,\ldots, p$ be the return of $p$ assets and $R$ the vector of $R_i$. We clearly see the expected return of $i$th asset $r_i = E(R_i)$ and risk $\sigma_i = \sigma_{R_i}$.
We have the covariance matrix of $R$ to be $\Sigma:=(\sigma_{ij})$.\\
A \textbf{portfolio} is a new asset consisting of existing assets
\[
R_N=\sum_{i=1}^p R_p = w^TR
\]
where $w_i$ is the weight of $i$th asset, and satisfies $\sum_{i=1}^p w_i = 1$.\\
The expected return of portfolio $E(R_N)=w^TE(R)$.\\
Variance of the portfolio is $w^T\Sigma w$.\\
The Sharpe ratio is
\[
\frac{E(R_N-r_F)}{\sigma_N}
\]
\end{definition}
\begin{theorem}[Portfolio of One Risky Asset And One Risk-free Asset]
\hfill\\\normalfont $R_N=wR_1+(1-w)r_f$. where $R_1$ has expected return $r_1$ and risk $\sigma_1$ and $r_f$ is the return of risk-free asset.\\
Under this case, we have
\begin{itemize}
  \item $r_N=wr_1 + (1-w)r_f$
  \item $\var(R_N)=w^2\sigma_1^2$
  \item Sharpe ratio is $\frac{r_1-r_f}{\sigma_1}$
  \item The line $(\sigma_N, r_N)$ is called the capital market line.
\end{itemize}
\end{theorem}
\begin{theorem}[Global Minimum Variance Portfolio]
\hfill\\\normalfont we have $w=\frac{\Sigma^{-1}\mathbf{1}}{\mathbf{1}^T\Sigma^{-1}\mathbf{1}}$.
\end{theorem}
\begin{theorem}[Tangency Portfolio]
\hfill\\\normalfont we have $w=\frac{\Sigma^{-1}(r-r_f\mathbf{1})}{\mathbf{1}^T\Sigma^{-1}(r-r_f\mathbf{1})}$
\end{theorem}
\begin{theorem}[Efficient Frontier]
\hfill\\\normalfont Suppose $w_t$ with target returns $r_t$ is efficient, then
\[
w_t=\frac{c-br_t}{ac-b^2}\Sigma^{-1}\mathbf{1}+\frac{ar_t-b}{ac-b^2}\Sigma^{-1}r
\]
where $a=1^T\Sigma^{-1}1$, $b=1^T\Sigma^{-1}r$ and $c=r^T\Sigma^{-1}r$.\\
Also, suppose $w_A, w_B$ efficient, then $w_C=\alpha w_A + (1-\alpha)w_B$ is also efficient as long as $r_C>r_{Min.Var}$.
\end{theorem}
\clearpage
\section{Copula}
\begin{definition}[Copula]
\hfill\\\normalfont A copula is a special multivariate cumulative distribution function(CDF), 
\[
C(u_1, \ldots, u_p)
\]
whose univariate marginal distribution are all $U(0,1)$.\\
As a result, copula has the following property:
\begin{itemize}
  \item $(u_1, \ldots, u_p)\in [0,1]^p$.
  \item $C(u_1,\ldots, u_p)$ is increasing on $[0,1]^p$.
  \item $C(u_1,\ldots, u_p)$ has margin CDF as
  \[
C_k(u)=C(1,\ldots, 1, u, 1,\ldots, 1) = u
  \]
  for all $u\in [0,1]$.
\end{itemize}
The density of a copula $c(u_1,\ldots, u_p)$ is given by
\[
c(u_1,\ldots, u_p) = \frac{\partial^p C(u_1,\ldots, u_p)}{\partial u_1 \ldots \partial u_p}
\]
\end{definition}
\begin{theorem}[Sklar's theorem]
\hfill\\\normalfont Any continuous random vector $X=(\mathbf{x}_1,\ldots, \mathbf{x}_p)^T$ has a copula.
\begin{itemize}
  \item If $\mathbf{x}$ is continuous univariate with cumulative distribution $F_{\mathbf{x}}(x)$, then 
  \[
F_{\mathbf{x}}(x)\sim U[0,1]
  \]
  \item For any random vector $X=(\mathbf{x}_1,\ldots, \mathbf{x}_p)^T$ with joint CDF $F(x_1,\ldots, x_p)$ and marginal CDF $F_1(x_1),\ldots, F_p(x_p)$, let $\mathbf{u}_i=F_i(\mathbf{x}_i)$, and $\mathbf{U}=(\mathbf{u}_1,\ldots, \mathbf{u}_p)^T$, then
  \begin{align*}
C^\ast(u_1,\ldots, u_p):&=P(\mathbf{u}_1\leq u_1, \ldots, \mathbf{u}_p\leq u_p)\\
&=P(\mathbf{x}_1\leq F_1^{-1}(u_1),\ldots, \mathbf{x}_p\leq F_p^{-1}(u_p))\\
&=F(F_1^{-1}(u_1), \ldots,F_p^{-1}(u_p))
  \end{align*}
  is a copula.\\
  Under such definition, we have $F(x_1,\ldots, x_p)=C^\ast(F_1(x_1),\ldots, F_p(x_p))$.
  \item For the copula above, its density is
  \[
c^\ast(u_1,\ldots, u_p)=\frac{f(F_1^{-1}(u_1), \ldots, F_p^{-1}(u_p))}{f_1(F_1^{-1}(u_1))\times\cdots\times f_p(F_p^{-1}(u_p))}
  \]
  \item For any marginal distribution $F_1(u_1),\ldots, F_p(u_p)$, we can introduce a copula $C_{new}(u_1, \ldots, u_p)$ to combine them, and generate a joint distribution
  \[
F_{new}(y_1, \ldots, y_p)=C_{new}\{F_{\mathbf{y}_1}(y_1), \ldots, F_{\mathbf{y}_p}(y_p)\}
  \]
  and density
  \[
f_{new}(y_1, \ldots, y_p) =c_{new}\{F_{\mathbf{y}_1}(y_1), \ldots, F_{\mathbf{y}_p}(y_p)\}f_{\mathbf{y}_1}(y_1)\times\cdots\times f_{\mathbf{y}_p}(y_p)
  \]
  \item $X$ and $Y$ are independent if and only if their copular is
  \[
C(u,v)=uv, 0\leq u,v\leq 1
  \]
  or
  \[
c(u,v)=1, 0\leq u,v\leq 1
  \]
\end{itemize}
\end{theorem}
\subsection{Commonly Used Copulas}
\begin{itemize}
  \item Independence copula: $C(u_1,\ldots, u_p)=u_1\cdots u_p$
  \item Co-monotonicity copula: $M(u_1,\ldots, u_p) = \min(u_1,\ldots, u_p)$
  \item Counter-monotonicity copula: $W(u_1,\ldots, u_p) = \max\{1-p+\sum_{i=1}^p u_i, 0\}$.
\end{itemize}
For any copula $C(u_1,\ldots, u_p)$, we have
\[
W\leq C\leq M
\]
\end{theorem}
\begin{definition}[Gaussian Copula]
\hfill\\\normalfont For a given covariance matrix $\Sigma\in\mathbb{R}^{p\times p}$, the Gaussian copula with parameter matrix $\Sigma$ and expectation $\mu$ can be written as
\[
C_{\Sigma}^{\text{gauss}}(u)=\Phi_{\Sigma}(\Phi_1^{-1}(u_1),\ldots, \Phi_p^{-1}(u_p))
\]
where $\Phi_{k}^{-1}$ is the inverse cumulative distribution of $N(0,\sigma_{kk})$ and $\Phi_{\Sigma}$ is the joint cumulative distribution function of a multivariante normal distribution with mean vector $0$ and covariance matrix equal to the correlation matrix $\Sigma = R$.\\
For bivariate case, we have
\[
C_\rho^{\text{gauss}}(u_1, u_2) = \Phi_\rho(\Phi_{1}^{-1}(u_1), \Phi_2^{-1}(u_2))
\]
or equivalently,
\[
C(u_1, u_2; \rho) = \int_{-\infty}^{\Phi^{-1}(u_1)}\int_{-\infty}^{\Phi^{-1}(u_2)}\frac{1}{2\pi\sqrt{1-\rho^2}}\exp\{-\frac{s_1^2-2\rho s_1s_2+s_2^2}{2(1-\rho^2)}\}\diff s_1\diff s_2
\]
where $\Phi^{-1}$ is inverse of CDF of standard normal.\\
Note this copula is jointly normal, even if the distribution is not normal.
\end{definition}
\begin{definition}[{$t$} copula]
\hfill\\\normalfont 
\[
C_{\nu, \Sigma}(u_1,\ldots, u_p)=\int_{-\infty}^{t_\nu^{-1}(u_1)}\cdots \int_{-\infty}^{t_\nu^{-1}(u_p)}\frac{\Gamma[(\nu+p)/2]}{\Gamma(\nu/2)(\nu\pi)^{p/2}}|\Sigma|^{-1/2}[1+\frac{1}{\nu}X^T\Sigma^{-1}X]^{-(\nu+p)/2}\diff x_1\cdots \diff x_p
\]
where $t_\nu^{-1}$ is the quantile function of a standard univariate $t_v$ distribution, and $X = (x_1,\ldots, x_p)^T$.
\end{definition}
\begin{theorem}
\hfill\\\normalfont The copula generated by $X=(x_1,\ldots, x_p)$ is the same as that generated by $\bar{X}:=(\frac{x_1-\mu_1}{\sigma_1}, \ldots, \frac{x_p-\mu_p}{\sigma_p})$
\end{theorem}
\begin{definition}[Archimedean Copula]
\hfill\\\normalfont $C(u_1,\ldots, u_p)=\psi(\psi^{-1}(u_1) +\cdots + \psi^{-1}(u_p))$ where $\psi$ is called \textit{generater}.\\
The table for famous generaters is omitted here.
\end{definition}
\begin{theorem}[Mixture of copula is still a copula]
\hfill\\\normalfont If $C$ and $D$ are two copulas, then for any $w\in[0,1]$, $wC+(1-w)D$ is still a copula.
\end{theorem}
\subsection{Tail Dependence of Bivariate Copula}
Let $I_i = \mathbf{1}(y_i\leq F_i^{-1}(u)), i = 1,2$. We can consider $I_i$ to be the indicator for default.\\
Then $E(I_1I_2) = F(F_1^{-1}(u), F_2^{-1}(u)) = C(u,u)$ is a copula.\\
Furthermore, we have
\[
Cor(I_1, I_2) = \frac{C(u,u)-u^2}{u-u^2}
\]
So the lower tail correlation of $y_1$ and $y-2$ is
\[
\lim_{u\to 0} Cor(I_1,I_2)=\lim_{u\to 0}\frac{C(u,u)}{u}
\]
\begin{definition}[Lower Tail Dependence, Upper Tail Dependence]
\hfill\\\normalfont We define lower tail dependence $\lambda_L$ 
\[
\lambda_L := \lim_{q\to 0^+}P(\mathbf{y}_2\leq F_{\mathbf{y}_2}^{-1}(q)\mid \mathbf{y}_1\leq F_{\mathbf{y}_1}^{-1}(q)) = \lim_{q\to 0^+}\frac{C_Y(q,q)}{q},\text{ where }Y=(\mathbf{y}_1,\mathbf{y}_2)^T
\]
and upper tail dependence $\lambda_U$
\[
\lambda_U:=\lim_{q\to 1^-}P(\mathbf{y}_2\geq F_{\mathbf{y}_2}^{-1}(q)\mid \mathbf{y}_1\geq F_{\mathbf{y}_1}^{-1}(q)) =\lim_{q\to 1^{-}}\frac{1-2q + C_Y(q,q)}{1-q}
\]
\end{definition}
The tail dependence of copula table is included in the lecture note and is omitted here.
\subsection{Empirical Copula}
Consider only bivariate case $Z=(\mathbf{x}, \mathbf{y})$, whose distribution, both marginal and joint) is unknown and needs to estimate.\\
For observations $(x_i, y_i), i = 1,\ldots, n$, the empricial CDF of $Z$ is 
\[
\hat{F}(x,y)=\frac{1}{n+1}\sum_{i=1}^n I(x_i\leq x, y_i\leq y)
\]
and the marginal empirical CDF is
\[
\hat{F}_\mathbf{x}(x)=\frac{1}{n+1}\sum_{i=1}^n I(x_i\leq x) = \frac{\#\{x_i: x_i\leq x\}}{n+1}
\]
and similarly
\[
\hat{F}_\mathbf{y}(y)=\frac{1}{n+1}\sum_{i=1}^n I(y_i\leq y) = \frac{\#\{y_i: y_i\leq y\}}{n+1}
\]
We define $u_j=\hat{F}_\mathbf{x}(x_j) = \frac{\text{rank of }x_j\text{ in }x_1,\ldots, x_n}{n+1}$ and
$v_j=\hat{F}_\mathbf{y}(y_j) = \frac{\text{rank of }y_j\text{ in }y_1,\ldots, y_n}{n+1}$, and they are observations from $\mathbf{u}_\mathbf{x}:=F_\mathbf{x}(\mathbf{x})$ and $\mathbf{u}_\mathbf{y}$. It is easy to see they can form a corpula.\\
We define the \textbf{empirical copula} to be
\[
\hat{C}(u_1,u_2)=\frac{1}{n}\sum_{i=1}^n I(u_i\leq u_1, v_i\leq u_2)
\]
\subsection{Fitting Copula}
To model a joint distribution of $(\mathbf{y}_1, \ldots, \mathbf{y}_p)$, we only need to model the marginal distribution of each variable $F_k(y_k\mid \theta_k), k = 1,\ldots, p$, as well as the copula density $c(u_1,\ldots, u_p\mid \theta_C)$.\\
For example, we can have $F(x,y) = C_t(t_{\nu_1}(\frac{x-\mu_1}{s_1}, t_{\nu_2}(\frac{x-\mu_2}{s_2})\mid \rho, \nu)$.\\
More generally, the joint density is $c(F_1(y_1\mid \theta_1), \ldots, F_p(y_p\mid \theta_p)\mid \theta_C)f_1(y_1\mid \theta_1)\cdots f_p(y_p\mid \theta_p)$. For samples, $Y_1 = (y_{11}, \ldots, y_{1p})^T, \ldots, Y_n=(y_{n1},\ldots, y_{np})^T$, and the log-likelihood is
\[
\log\{L\}=\sum_{i=1}^n \log[c(F_1(y_1\mid \theta_1), \ldots, F_p(y_p\mid \theta_p)\mid \theta_C)f_1(y_1\mid \theta_1)\cdots f_p(y_p\mid \theta_p)]
\]
The maximum values for the parameters are the MLE of the parameters $(\theta_1, \ldots, \theta_p; \theta_C)$.\\
Note that $F_k(y_{ik}\mid \theta_k)\approx u_{ik}$, so we can estimate $\theta_C$ directly by maximizing 
\[
\log\{L\}=\sum_{i=1}^n \log\{c(u_{ik}, \ldots, u_{ik}\mid \theta_C)\}
\]
If we need to estimate the distribution function, we then only need to estimate each marginal density separately, by maximizing 
\[
\sum_{i=1}^n \log\{f_k(y_{ik}\mid \theta_k)\}
\]
We can also use AIC or BIC to select among different copulas. We prefere the copula with smallest AIC or BIC.
\subsection{Monte Carlo Simulation}
Suppose $G(u), F(x_1,\ldots, x_p)$ are strictly continuous increasing CDFs, and $F_i(x_i), i = 1,\ldots, p$ are the marginal CDF of $F(x_1,\ldots, x_p)$.
Suppose $\mathbf{z}\sim G(x)$, and let $\mathbf{u}=G(\mathbf{z})$, then
\[
\mathbf{u}\sim U(0,1)
\]
If $\mathbf{u}\sim U(0,1)$, and let $\mathbf{z}=G^{-1}(u)$, then
\[
\mathbf{z}\sim G(z)
\]
Let $X=(\mathbf{x}_1, \ldots, \mathbf{x}_p)\sim F(x_1,\ldots, x_p)$, then
\begin{itemize}
  \item Marginal $\mathbf{x}_i\sim F_i(x_i)=F(\infty, \ldots, x_i,\ldots, \infty)$.
  \item Let $U=(F_1^{-1}(\mathbf{x}_1),\ldots, F_p^{-1}(\mathbf{x}_p))$, then
  \[
U\sim F(F_1^{-1}(u_1),\ldots, F_p^{-1}(u_p))
  \]
\end{itemize}
Let $U=(\mathbf{u}_1, \ldots, \mathbf{u}_p)\sim C(u_1,\ldots, u_p)$, and let $X=(F_1^{-1}(\mathbf{u}_1), \ldots, F_p^{-1}(\mathbf{u}_p))$.\\
For a general distribution $\mathbf{x}\sim F(x)$, we can generate random numbers from it
\[
\mathbf{x}_i = F^{-1}(\mathbf{u}_i)
\]
where $\mathbf{u}_i\sim U(0,1)$.\\
As a result, 
\[
E(\mathbf{x}^k)=\int x^k\diff F(x) = \lim_{N\to \infty}\frac{\sum_{i=1}^N \mathbf{x}_i^k}{N}
\]
and
\[
\VaR_q(X)=-Q_q(\mathbf{x})=-\lim_{N\to\infty} \textit{quantile}(\{x_i, i = 1, \ldots, N\}, \textit{probs} = q)
\]
and ES
\[
\ES_q(\mathbf{x})= \frac{1}{q}\int_{-\infty}^{Q_q(\mathbf{x})}x\diff F(x)=-\lim_{N\to\infty}\frac{\sum_{\mathbf{x}_i; \mathbf{x}_i < Q_q(\mathbf{x})}\mathbf{x}_i}{\#\{\mathbf{x}_i:\mathbf{x}_i< Q_q(\mathbf{x})\}}
\]
\clearpage
\section{CAPM}
\subsection{Sharpe's Single Index Model}
Suppose $N$ assets denoted by $i = 1,\ldots, N$, with returns in period $t$, $R_{it}, t = 1,\ldots, T$
\[
R_{it} = \alpha_i + \beta_iR_{mt} + \epsilon_{it}, i = 1,\ldots, N; t=1,\ldots, T
\]
where
\begin{itemize}
\item $\alpha_i,\beta_i$ constant over time,
\item $R_{mt}$ is return on \textbf{diversified} market index
\item $\epsilon_{it}$ random error term unrelated to $R_{mt}$.
\end{itemize}
We also assume
\begin{itemize}
  \item $\cov(R_{mt}, \epsilon_{is}) = 0$ for all $i, t, s$.
  \item $\cov(\epsilon_{is}, \epsilon_{jt})$ for all $i\neq j, t$ and $s$
  \item $\epsilon_{it}\sim IID N(0, \sigma_{\epsilon, i}^2)$
  \item $R_{m,t}\sim IID N(\mu_m, \sigma^2 m)$
\end{itemize}
We can easily see that, by computing the numerator
\[
\beta_i = \frac{\cov(R_{it}, R_{mt})}{\cov(R_{mt})} = \frac{\sigma_{iM}}{\sigma_m^2}
\]
which is the contribution of asset $i$ to the volatility of the market index.\\
Also, we notice that returns is correlated only through their exposure to common market-wide news.\\
\begin{theorem}[Statistical Properties of SI Model]
For SI Model $R_{it} = \alpha_i + \beta_i R_{mt} + \epsilon_{it}$,
\begin{itemize}
  \item $\mu_i = \alpha_i + \beta_i + \mu_m$
  \item $\sigma_i^2 = \beta_i^2\sigma_m^2 + \sigma_{\epsilon, i}^2$
  \item $\sigma_{ij} = \cov(R_{it}, R_{jt}) = \sigma_m^2\beta_i\beta_j$.
  \item $R_{it}\sim N(\mu_i, \sigma_i^2)$.
\end{itemize}
\end{theorem}
\begin{theorem}[Decomposition of Total Variance]
\[
\sigma_i^2 = \beta_i^2\sigma_m^2 + \sigma_{\epsilon, i}^2
\]
and we have
\[
1 = \frac{\beta_i^2\sigma_m^2}{\beta_i^2} + \frac{ \sigma_{\epsilon, i}^2}{\sigma_i^2} = R_i^2 + (1 - R_i^2)
\]
where $R_i^2$ is proportion of market variance, and $1-R_i^2$ proportion of non-market variance.\\
Similarly, covariance matrix can be decomposed
\[
\Sigma = \sigma_m^2\mathbf{\beta}^T\mathbf{\beta} + \text{Diag}[\sigma_{\epsilon, 1}^2 \ldots \sigma_{\epsilon, n}^2]
\]
\end{theorem}
\subsection{CAPM}
The capital asset pricing model is 
\[
R_i = r_f + \beta_i(R_m - r_f) + \epsilon_i
\]
where 
\begin{itemize}
  \item $R_m$ is return of the market
  \item $E(R_m)-r_f$ is known as market premium.
  \item $E(R_i)-r_f$ is known as risk premium.\\
  CAPM can be written as $E(R_i)-r_f = \beta_i(E(R_m)-r_f)$
  \item $\beta_i$ is the sensitivity of expected excess asset returns to the expected excess market returns, or
  \[
\beta_i = \frac{\cov(R_i, R_m)}{\cov(R_m)}
  \]
  \item We can also write $\beta_i$ as $\beta_i = \rho_{R_i, R_m}\frac{\sigma_{R_i}}{\sigma_{R_m}}$, where $\rho$ is the correlation coefficient.
  \item The expected market return is usually estimated by measuring the log-return of the historical returns.
\end{itemize}
To price using CAPM, notice
\[
E(R_T) = \frac{E(P_T)-P_0}{P_0}
\]
Here, $P_0$ is fixed but $P_T$ is a random variable.\\
By CAPM,
\[
E(R_T) = r_f + \frac{\cov(R_T, R_m)}{\cov(R_m)}(E(R_m)-r_f)
\]
So,
\[
P_0 = \frac{1}{1+r_f}[E(P_T) - \frac{\cov(P_T, R_m)}{\cov(R_m)}(E(R_m)-r_f)]
\]
where $P_T$ is the expected price of asset.
\subsection{Derivation of CAPM}
Suppose $p$ risky asset $R_i$ with expected return $r_i$ and risk $\sigma_i$, $i = 1,\ldots, p$. Let $r_f$ be risk free return.\\
The optimal market portfolio
\[
R_m = \tilde{w}_iR_1 + \ldots + \tilde{w}_pR_p
\]
is the tangency portfolio, or at the CML, with expected return ,risk and covariance
\[
r_m:=E(R_m), \sigma_m^2 = \cov(R_m), \sigma_{i, m}= \cov(R_i, R_m)
\]
Consider a portfolio,
\[
R_N = \sigma_{i=1}^p w_iR_i
\]
with expected return
\[
r_N = \sum_{i=1}^p w_iE(R_i)
\]
Obviously,
\[
\sigma_N^2 = \sum_{i=1}^p w_i^2 \sigma_i^2 + \sum_{i}\sum_{j\neq i} w_i w_j \cov(R_i, R_j)
\]
We define
\[
C = \sigma_N + \lambda[r_N - \sum_{i=1}^p w_iE(R_i)-\underbrace{(1-\sum_{i=1}^p w_i)r_f}{=0}]
\]
We need to minimize $C$. Using Lagrange multiplier, we have
\[
\sigma_N = \lambda(r_N - r_f)
\]
where $\lambda = \frac{r_m-r_f}{\sigma_m}$. Then we use Lagrange multiplier equation to get the equation on $E(R_i)$, which gives CAPM.
\subsection{MultiFactor Models}
Suppose asset return $R_i$ is driven by $K$ common factors, $f_1, \ldots, f_K$ and idiosyncratic noise $u_i$. A multi-factor regression model is
\[
R_i = \alpha_i + \sum_{j=1}^K \beta_{ij}f_j + u_i
\]
for $i = 1, \ldots, n$, a total of $n$ assets.\\
Here, $\alpha_i$ is regression intercept for return of asset $i$.\\
$f_j$ are common factors driving \textbf{all} asset returns, with $\cov(f_i)=\lambda_i$, and $\cov(f_i, f_j)=0$ for $i\neq j$.\\
$\beta_{ij}$ gives how sensitive the return of asset $i$ is, with respect to $j$th factor, which is called the factor loading of asset $i$ on factor $f_j$.\\
$u_i$ is idiosyncratic component in $i$th return, with $E(u_i)=0$ and $\cov(u_i)=\sigma_{u_i}^2$, and $\cov(u_i,u_j)=0$ for $i\neq j$.\\
$u_i$ is independent of $f_1, \ldots, f_K$.\\
\begin{theorem}\normalfont If market is efficient, and factors are indeed complete, the the model is
\[
R_i-r_f = \sum_{j=1}^K \beta_{ij}f_{j}' + u_i
\]
where $f_j'$ needs to remove risk-free return.\end{theorem}
\begin{theorem}[Expectation and Variance of {$R_i$}]
\hfill\\\normalfont 
\[
E(R_i)-r_f = \sum_{j=1}^K \beta_{ij}E(f_j')
\]
\[
\cov(R_i) = \sum_{j=1}^K \beta_{ij}^2\cov(f_j) + \sigma_{u_i}^2
\]
\[
\cov(R_i, R_j) = \sum_{k=1}^K \beta_{ik}\beta_{jk}\cov(f_k)
\]
\end{theorem}
\begin{definition}[Fama-French Model]
\hfill\\\normalfont 
\[
R_{i,t} = r_{f,t} + \beta_{i,m}(R_{t,m}-r_{f,t})+\beta_{i,s}SMB_t + \beta_{i,v}HML_t + \alpha_i + \epsilon_{i,t}
\]
where
\begin{itemize}
  \item $R_{i,t}$ the stock return during period $t$.
  \item $r_{f,t}$ the return on risk-free asset.
  \item $\alpha_i$ zero, under certain assumptions.
  \item $\beta$ is beta
  \item $\epsilon_{i,t}$ regression error.
  \item $SMB$ is the return of small capital firms minus that of big.
  \item $HML$ is return of high (book value)/(market value) ratio minus that of low ratio
\end{itemize}
\end{definition}
\begin{definition}[Arbitrage Pricing Model]
\hfill\\\normalfont
\[
E(r_j) = r_f + \sum_{k=1}^K \beta_{jk}RP_k
\]
where $RP_k$ is the risk premium of the factor, ususally $RP_k=E(f_k-r_f)$, or $=E(f_k)$.
\end{definition}
\subsection{CAPM and Factor Models: Principal Component Analysis}
Given a $p$ dimensional random variable $X=(x_1,\ldots, x_p)^T$, with $E(X)=\mu$ and $\Sigma=\cov(X)$.\\
Consider 
\[
\Sigma=W\begin{pmatrix}\lambda_1 & 0 & \cdots & 0 \\ 0 & \lambda_2 & \cdots & 0 \vdots &&& \\ 0 & 0 & \cdots &\lambda_p\end{pmatrix} W^T
\]
which is the eigenvalue-eigenvector decomposition, where $\lambda_1\geq \ldots\geq \lambda_p\geq 0$, and $W=(\mathbf{w}_1,\ldots, \mathbf{w}_p)$ is an orthogonal matrix, i.e. $W_W^T = W^TW=I_p$.\\
The orthogonal means $\mathbf{w}_i\mathbf{w}_j = 1$ if and only if $i=j$, and $0$ otherwise.\\
Also, we have $\Sigma \mathbf{w}_i = \lambda_i\mathbf{w}_i$.\\
We can write $\Sigma = \lambda_1 \mathbf{w}_1\mathbf{w}_1^T + \ldots + \lambda_p \mathbf{w}_p\mathbf{w}_p^T$, and
\[
\mathbf{w}_i^T\Sigma\mathbf{w}_i = \lambda_i
\]
We can $\mathbf{z}_i = \mathbf{w}_i^TX$, the $i$th principal component, where
\[
\cov(\mathbf{z}_i) = \lambda_i
\]
and
\[
\cov(\mathbf{z}_i, \mathbf{z}_j) = 0 \text{ if }i\neq j
\]
The first principal component of $X$ is the linear combination $\mathbf{z}_1 = \mathbf{w}_1^T X$ that maximizes $\cov(\mathbf{z}_1)$ subject to the constraint $\mathbf{w}_1^T\mathbf{w}_1 = 1$.\\
Similarly, $\mathbf{z}_2$ maximizes $\cov(\mathbf{z}_2)$ subject to $\mathbf{w}_2^T\mathbf{w}_2=1$ and $\cov(\mathbf{z}_2, \mathbf{z}_1) = 0$.\\
This applies to all other $\mathbf{z}_i$'s.\\
We can write $(\mathbf{z}_1,\ldots, \mathbf{z}_p)^T = (\mathbf{w}_1,\ldots, \mathbf{w}_p)^TX$.\\
For total variance $\sum_{i=1}^p \cov(x_i)$, we have
\begin{align*}
\sum_{i=1}^p \cov(x_i) &= tr(\Sigma) = tr(W\begin{pmatrix}\lambda_1 & 0 & \cdots & 0 \\ 0 & \lambda_2 & \cdots & 0 \vdots &&& \\ 0 & 0 & \cdots &\lambda_p\end{pmatrix} W^T)\\
&=tr(\begin{pmatrix}\lambda_1 & 0 & \cdots & 0 \\ 0 & \lambda_2 & \cdots & 0 \vdots &&& \\ 0 & 0 & \cdots &\lambda_p\end{pmatrix})\\
&=\sigma_{i=1}^p \lambda_i
\end{align*}
And $\frac{\lambda_i}{\sum_{j=1}^p \lambda_j}$ is the proportion of total variance in $X$ explained by the $i$th principal component.\\
Suppose we define $z=w^TX$, and use it to predict each variables in $X$ by model
\[
x_i = \alpha_i + \beta_i\mathbf{z} + \epsilon_i
\]
where $E(\epsilon_i) = 0$ and $\cov(z, \epsilon_i) = 0$. The prediction error for this stock $i$ is defined as
\[
\epsilon_i = x_i - \alpha_i -\beta_iz
\]
We hope to minimize
\[
\min_w E(\sum_{i=1}^p\epsilon_i^2)
\]
then the solution is $w=\mathbf{w}_1$ and common factor is $\mathbf{z}=\mathbf{w}_1^TX$. where $\mathbf{w}_1$ is the first principal component.
\clearpage
\section{Financial Time Series}
\subsection{Theoretical definitions of Time Series}
\begin{definition}[Time Series, Realization]
\hfill\\\normalfont A \textbf{time series} $\{Z_t\}$ is a sequence of random variables indexed by time $t$
\[
\{\cdots, Z_1, \cdots, Z_{t-1}, Z_t, Z_{t+1},\cdots\}
\]
A realization of a stochastic process is the sequence of observed data
\[
\{\cdots, Z_1=z_1,\cdots, Z_{t-1}=z_{t-1},Z_t=z_t, Z_{t+1}=z_{t+1},\cdots\}
\]
We use $\{z_t\}$ for both theoretical time series and its observations.
\end{definition}
\begin{definition}[Mean, Variance and Covariance Function]
\hfill\\\normalfont Let $\{z_t\}$ be a time series with $E(z_t^2)<\infty$.\\
The mean function is $\mu_t = E(z_t)$.\\
The variance function is $\sigma_t^2 = \cov(z_t) = E[(z_t-\mu_t)^2]$\\
The covariance function is $\gamma(r,s)=\cov(z_r, z_s)=E[(z_r-\mu_r)(z_s-\mu_s)]$
\end{definition}
\begin{definition}[Strictly Stationary]
\hfill\\\normalfont $\{z_t\}$ is strictly stationary if for any given finite integer $k$, and for any set of subscript $t_1,\ldots, t_k$, the joint distribution of $z_{t_1}, \ldots, z_{t_k}$ depends only on $t_1-t_2$, \ldots, $t_{k-1}-t_k$, but not directly on $t_1,\ldots, t_k$.
\end{definition}
Some of the consequences are
\begin{itemize}
  \item If $\{Z_t\}$ is an IID sequence, then it is strictly stationary.
  \item Let $\{Z_t\}$ be an iid sequence and let $X$ independent of $\{Z_t\}$. Let $Y_t = Z_t + X$, and then $\{Y_t\}$ is strictly stationary.
  \item For any function $g$, $\{g(z_t)\}$ is also strictly stationary.
\end{itemize}
\begin{definition}[Weakly Stationary]
\hfill\\\normalfont $\{z_t\}$ is weakly stationary if 
\begin{itemize}
  \item $\mu_t$ does not depend on $t$, denoted by $\mu$
  \item $\gamma(r,s)=\gamma(|r-s|)$. 
\end{itemize}
\end{definition}
By the definition, we have $\cov(z_t)=0$.
\begin{theorem}\normalfont If $\{Z_t\}$ is strictly stationary and $\cov(Z_t)<\infty$, then $\{Z_t\}$ is also weakly stationary. However, a weakly stataionary time series is usually not strictly stationary.
\end{theorem}
\begin{definition}[Autocovariance Function]
\hfill\\\normalfont If $\{z_t\}$ is stationary, then
\begin{itemize}
\item $\gamma(h)=\cov(z_{t+h}, z_t)$ is called autocovariance function(ACVF) of $\{z_t\}$.
\item $\rho_z(h):=\frac{\gamma(h)}{\gamma(0)}=\Corr(z_{t+h}, z_t)$ is called \textbf{autocorrelation} function(ACF) at lag $h$ of $\{z_t\}$.
\end{itemize}
Basic properties of ACVF include
\begin{itemize}
  \item $\gamma(0)\geq0$
  \item $|\gamma(h)|\leq \gamma(0)$ for all $h$
  \item $\gamma(h)=\gamma(-h)$
\end{itemize}
\end{definition}
\begin{definition}[White Noise Sequence]
\hfill\\\normalfont If $\{z_t\}$ is a stationary with $E(z_t)=0$, $\gamma(0)=\sigma^2$ and
\[
\gamma(h) = 0, \text{ for any }h\neq 0
\]
then we call $\{z_t\}$ white noise sequence, denoted by $z_t WN(0,\sigma_z^2)$.
\end{definition}
Suppose $\{y_t\}$ stationary, and $\{\epsilon_t\}$ a $WN(0, \sigma^2)$. If they are independent, i.e. $E(y_t\epsilon_{t+k})=E(y_t)E(\epsilon_{t+k})$ for all $t, k$, then
\[
z_t=(a+by_t^2)^{1/2}, a\geq 0, b\geq 0
\]
is also a WN.\\
Suppose $\{\epsilon_t\}$ is a sequence of IID $N(0,\sigma^2)$, thus WN. Let
\[
z_t=(a+bz_{t-1}^2)^{1/2}\epsilon_t, a\geq 0, b\geq 0
\]
then $\{z_t\}$ is also a WN. THis is the Arch model.\\
We can build stationary sequences from White Noise. For example, $y_t = z_t+\theta z_{t-1}$, where $\{z_t\}$ is a white noise.
\begin{theorem}[Wold's Decomposition Theorem]
\hfill\\\normalfont Any weakly stationary time series $\{Y_t\}$ can be represented in the form
\begin{align*}
Y_t &= \mu + \epsilon_t + \phi_1\epsilon_{t-1} + \phi_2\epsilon_{t-2} + \cdots\\
    &=\mu + \sigma_{k=0}^\infty \phi_k\epsilon_{t-k}
\end{align*}
where
\[
\phi_0 = 1, \sigma_{k=1}^\infty \phi_{k}^2 <\infty
\]
and
\[
\{\epsilon_t\}\text{ is }WN(0, \sigma^2)
\]
\end{theorem}
Such $Y_t$ has the following properties:
\begin{itemize}
  \item $E(Y_t)=\mu$.
  \item $\gamma(0)=\cov(Y_t)=\sigma^2\sum_{k=0}^\infty \phi_k^2$
  \item When $h>1$, $\gamma(h)=\sigma^2\sum_{j=0}^\infty \phi_j\phi_{h+j}$.
\end{itemize}
In finance data, the stationary can usually be obtained by considering the first-difference
\[
z_t = x_t  - x_{t-1}
\]
If $z_t$ is stationary, then $x_t$ is integrated of order 1.\\
We can also consider $k$th-difference.
\subsection{Sample ACVF, ACF}
Suppose we have observation $z_1,\ldots, z_n$. Suppose that $\{z_t\}$ is stationary, we then have to estimate $\mu_z$, $\gamma(h)$ and $\rho_z(h)$, with $h= 0,1,\ldots$.\\
We have
\begin{itemize}
  \item Sample Mean $\bar{z}=\frac{1}{n} \sum_{t=1}^n z_t$
  \item Sample Variance $\hat{\sigma}_z^2 = \frac{1}{n}\sum_{t=1}^n (z_t-\bar{z})^2 = \hat{\gamma}(0)$
  \item Sample autocovariance function at lag $h$ (SACVF):
  \[
\hat{\gamma}(h)=\frac{1}{n}\sum_{t=1}^{n-h} (z_t-\bar{z})(z_{t+h}-\bar{z})
  \]
  \item Sample autocorrelation coefficient function at lag $h$, (SACF):
  \[
\hat{rho}(h) = \frac{\hat{\gamma}(h)}{\hat{\gamma}(0)}
  \]
\end{itemize}
\begin{theorem}[Testing {$\rho(h)=0$}]
\hfill\\\normalfont Suppose $z_1,\ldots, z_n$ is realization of stationary time series.\\
The SACF at lag $h$ is
\[
r_h=\frac{\sum_{t=1}^{n-h} (z_t-\bar{z})(z_{t+h}-\bar{z})}{\sum_{t=1}^n (z_t-\bar{z})^2}
\]
where
\[
\bar{z}=\frac{1}{n}\sum_{i=1}^n z_t
\]
Under $H_0': \rho(h)=0$ for all $h>0$, 
\[
s_{r_h}=(\frac{1}{n})^{\frac{1}{2}}
\]
If $|r_h|<2s_{r_h}$, then accept $H_0$. Otherwise, reject $H_0$.
\end{theorem}
\begin{theorem}[Test of White Noise]
\hfill\\\normalfont 
The Ljung-Box test has test statistics 
\[
Q(h)=n(n+2)\sum_{k=1}^h \frac{r(k)^2}{n-k}
\]
where $n$ is sample size, $r(k)$ is sample autocorrelation at lag $k$, and $h$ the number of lags being tested.\\
Under $H_0: \rho(k)=0, k = 1,\ldots, h$,
\[
Q(h)\sim \xi^2(h)
\]
For significance level $\alpha$, the critical region for rejection of hypothesis of randomness is rejected if the statistic 
\[
Q^\ast(h)>\xi_{1-\alpha}^2(h)
\]
where $\xi^2_{1-\alpha}(h)$ is the $(1-\alpha)$-quantile of the chi-square distribution with $h$ degrees of freedom.\\
\[
p\text{-value} = P(\xi^2(h)>Q^\ast(h))
\]
We accept $H_0$ if $p$-value$>\alpha$.
\end{theorem}
\subsection{Moving Average Model}
For any integer $q\geq 1$, the moving average model of order $q$, $MA(q)$ is
\[
x_t = \mu + a_t +\theta_1a_{t-1} + \cdots + \theta_q a_{t-q}
\]
More generally, let 
\[
x_t = \mu + \sum_{j=0}^\infty \psi_ja_{t-j}
\]
where $\psi_0 = 1$, and $\{a_t\}\sim WN(0,\sigma^2)$ and
\[
\sum_{j=0}^\infty|\psi_j|<\infty
\]
$\{x_t\}$ is called a stationary general linear process or a $MA(\infty)$ process.\\
For $MA(q)$ we have
\begin{itemize}
  \item $E(x_t) = \mu$
  \item $\cov(x_t)=(1+\theta_1^2 + \cdots + \theta_q^2)\sigma^2$
  \item If $|h|\leq q$,
  \[
\cov(x_t, x_{t-h})=\sigma^2 \sum_{j=0}^{q-|h|} \theta_j\theta_{j+|h|}
  \]
  \item $\cov(x_t, x_{t-h})=0$ if $|h|>q$
\end{itemize}
\subsection{Autoregressive Model}
An autoregressive model of order $p$, $\AR(p)$ is
\[
x_t = \phi_0 + \phi_1x_{t-1} + \cdots + \phi_px_{t-p}+a_t
\]
where $a_t\sim WN(0, \sigma^2)$.\\
It follows
\[
E(x_t) = \phi_0 + \phi_1E(x_{t-1}) + \cdots + \phi_pE(x_{t-p})
\]
If it is stationary, let $\mu = E(x_t)$. We have
\[
\mu = \frac{\phi_0}{1-\phi_1-\cdots-\phi_p}
\]
So $\AR(p)$ can be written as
\[
x_t = (1-\phi_1-\cdots -\phi_p)\mu + \phi_1x_{t-1} + \cdots + \phi_px_{t-p} + a_t
\]
i.e.
\[
y_t = \phi_1 y_{t-1} + \cdots + \phi_py_{t-p} + a_t
\]
where $y_i = x_i - \mu$.
Therefore, in theoretical analysis, we only need to consider $\mu = 0$.\\

Consider $\AR(1)$, where $x_t\phi_1x_{t-1} + a_t$. If $|\phi_1| < 1$, we have
\[
x_t = \sum_{k=0}^\infty \phi_1^ka_{t-k}
\]
which is a general linear process with $\sum_{m=0}^\infty |\phi_1^m|<\infty$.\\
Therefore,
\begin{itemize}
  \item If $|\phi_1|<1$, $\AR(1)$ is stationary.
  \item If $|\phi_1|>1$, the time series will diverge to $\infty$.
  \item If $|\phi_1| = 1$, the times series will stay in a reasonable region, and $x_t = x_{t-1} + a_t$. This is not stationary, but a random walk.
\end{itemize}
\begin{definition}[Backshift Operator]
\hfill\\\normalfont For any time series $y_t$, define
\[
Ly_t = y_{t-1}, L^ky_t = y_{t-k}, k\geq 1
\]
A general $\AR(p)$ model can be written as
\[
\phi_p(L)y_t = a_t
\]
where $\phi_p(L)=1-\phi_1L - \cdots - \phi_pL^p$.
\end{definition}
A general $MA(q)$ model can be written as
\[
y_t = \theta_q(L)a_t
\]
where $\theta_q(L) = 1+ \theta_1L + \cdots + \theta_qL_q$.\\
Note that $(1-L)X_t = X_t - X_{t-1}$, which indicates that $1_L$ is a difference operator. We can also consider higher order difference $(1-L)^d$.\\
We have proved that for $\AR(1)$, if $|\phi_1| < 1$, then $x_t$ is stationary. Equivalently, if the root of 
\[
1-\phi_1L=0
\]
is outside the unit circle $|z|^2 = 1$,i,e, $|L|=|1/\phi_1>1$, then $\AR(1)$ modek is stationary.\\
For $\AR(p)$ model, if all roots of $\phi_p(L)=0$ are outside of unit circle, then the time series is stationary.\\
Generally, if $\AR(p)$ 
\[
y_t = \phi_0 + \phi_1y_{t-1} + \cdots +\phi_py_{t-p} + a_t
\]
is stationary, then it can be written as
\[
y_t = \tilde{\mu} + a_t + \psi_1a_{t-1} + \psi_2a_{t-2} + \cdots
\]
where $\sum_{k=1}^\infty |\psi_k| < \infty$. Thus,
\begin{itemize}
  \item $\cov(y_t, a_t) = \sigma^2$ and $\cov(y_t, a_{t+h}) = 0$ for any $h>0$
  \item $\cov(y_t, y_{y-h})=\phi_1\cov(y_{t-1}, y_{t-h}) + \cdots + \phi_p \cov(y_{t-p}, y_{t-h})$ when $h>0$
  \item $\cov(y_t, y_t) = \phi_1\cov(y_{t-1}, y_t) + \cdots + \phi_p\cov(y_{t-p}, y_t) + \sigma^2$
\end{itemize}
\begin{definition}[ARMA model, ARIMA model]
The $\ARMA(p,q)$ model is
\[
X_t = \phi_0 + \phi_1X_{t-1} + \cdots + \phi_pX_{t-p} + a_t + \psi_1 a_{t-1} + \cdots + \psi_q a_{t-q}
\]
Again, let $\mu = E(X_t)$, we have
\[
X_t-\mu = \phi_0 + \phi_1(X_{t-1}-\mu) + \cdots + \phi_p (X_{t-p}-\mu) + a_t + \psi_1a_{t-1} + \cdots + \psi_q a_{t-q}
\]
Sometimes, we consider $Z_t = (1-L)^d X_t$. If $Z_t$ follows $\ARMA(p,q)$ model, then we call $X_t$ follows $\ARIMA(p, d, q)$ model.
\end{definition}
\subsection{Parameter Estimation and Order Determination}
The $\ARMA(p,q)$ model has parameter 
\[
\theta = (\mu, \phi_1,\ldots, \phi_p, \psi_1,\ldots, \psi_q, \sigma^2)
\]
to be estimated. \\
If $p, q$ konwn, we can use maximum log-likelihood. If $p,q$ unknown, we can select them by AIC and BIC, where $D=p+ q + 1$.\\
For models, we also check the residuals(innovation): if the model is appropriate, the innovation should be white noise, which means there is no information left in the innovations for linear prediction.
\begin{definition}[Residuals]
\hfill\\\normalfont The residuals are
\[
\hat{a}_t = x_t -\{\hat{\mu} + \hat{\phi}_1(x_{t-1}-\hat{\mu} + \cdots + \hat{\phi}_p(x_{t-p}-\hat{\mu}) + \hat{\psi}_1\hat{a}_{t-1} + \cdots + \hat{\psi}_q\hat{a}_{t-q}\}
\]
And we can use Ljung-Box test.
\end{definition}
\subsection{Prediction of Stationary Process}
Let $\{y_t\}$ be a stationary process, e.g. $\ARMA(p,q)$ process with Wold representation
\[
y_t = \mu + a_t + \phi_1 a_{t-1} + \phi_2 a_{t-2} + \cdots
\]
For \textbf{prediction without any information}, the prediction of $y_{t+1}$ is
\[
E(y_{t+1}) = \mu + E(a_{t+1}) + \phi_1E(a_{t}) + \cdots = \mu
\]
For $\AR(p)$ model,
\[
E(y_{t+1}) = \phi_0 + \phi_1E(y_t) + \cdots + \phi_pE(y_{t-p+1}) + E(a_{t+1}) = \mu
\]
For \textbf{prediction with information up to time} $t$, let $I_t = \{y_t, y_{t-1}, \ldots\}$ be information up to time $t$. Our purpose is to predict $y_{t+h}, h\geq 1$ based on $I$.\\
A linear predictor is
\begin{align*}
&E(y_{T+h\mid t}\mid a_{T}, a_{T-1}, \ldots)\\
=&\mu + E(a_{T+h}) + \phi_1E(a_{T+h-1}) + \cdots + \phi_{h-1}E(a_{T+1}) + \phi_hE(a_T) + \phi_{h+1}E(a_{T-1}) + \cdots\\
=&\mu + \phi_ha_T + \phi_{h+1}a_{T-1} + \cdots
\end{align*}
\begin{definition}[Forecast Error]
\hfill\\\normalfont Define $y_{t+h\mid t}$ as forecast of $y_{t+h}$ based on $I_t$. The forecast error is
\[
e_{t+h\mid t} = y_{t+h}-y_{t+h\mid t}
\]
The mean squared error(MSE) of forecast is
\[
MSE(a_{t+h\mid t}) = E[(y_{t+h}-y_{t+h\mid t})^2]
\]
The forecast error of the best linear predictor is
\[
e_{t+h\mid t} = y_{t+h}-y_{t+h\mid t} = a_{t+h} + \phi_1 a_{t+h-1} + \cdots \phi_{h-1}a_{t+1}
\]
The MSE of the forecast error is
\[
MSE(a_{t+h\mid t}) = \sigma^2 (1 + \phi_1^2 + \cdots + \phi_{h-1}^2)
\]
If $\{a_t\}$ gaussian then
\[
y_{t+h\mid I_t}\sim N(y_{t+h}, \sigma^2 ( 1+ \phi_1^2 + \cdots + \phi_{h-1}^2))
\]
The 95\% confidence interval for $h$-step ahead prediction has the form 
\[
y_{t+h\mid t} \pm 1,96 \sqrt{\sigma^2 (1 + \phi_1^2 + \cdots + \phi_{h-1}^2)}
\]
And standard error of prediction
\[
s.e. = \sqrt{\sigma^2 (1 + \phi_1^2 + \cdots + \phi_{h-1}^2)}
\]
\end{definition}
\clearpage
\section{Conditional Heteroscedastic Model}
For time series $\{y_t\}$, denote information up to time $t$ by $I_t = \sigma\{y_t, \cdots\}$.\\
The conditional mean $\mu_t = E(y_t\mid I_{t-1})$ can be used to predict $y_t$ based on past information $I_{t-1}$.\\
Conditional variance, or \textbf{volatility} is $\sigma^2_t = \cov(yz_t\mid I_{t-1}) = E[(y_t-\mu_t)^2\mid I_{t-1}]$.\\
We can calculate volatility from high-frequency data, from implied volatility, or from weighted average.
\begin{theorem}[Weighted Average]
\hfill\\\normalfont Suppose the return $u_i$ has mean 0, or we consider $u_i-E(u_i)$.\\
We estimate the variance by
\[
\sigma_t^2 = \sum_{i=1}^m w_i u_{t-i}^2\text{ for prediction}
\]
or
\[
\sigma_t^2 = \sum_{i=0}^{m-1}w_i u_{t-i}^2\text{ for fitting}
\]
where $m$ can be infinity providing $\sum_{w_i}=1$.\\
A special case is
\[
w_k = (1-\lambda)\lambda^{k-1}, k = 1,\ldots
\]
and
\[
\sigma_t^2 = (1-\lambda)\sum_{i=1}^\infty \lambda^{i-1}u_{t-i}^2
\]
This leads to 
\[
\sigma_t^2 = (1-\lambda)u_{t-1}^2 + \lambda \sigma_{t-1}^2
\]
\end{theorem}
\begin{theorem}[GARCH Model]
\hfill\\\normalfont GARCH model has the following recurrence relation
\[
\sigma^2_t = \omega + \alpha u_{t-1}^2 + \beta \sigma^2_{t-1}
\]
with $\omega\geq 0, \alpha \geq 0, \beta \geq 0$, but $\alpha+\beta\leq1$.
\end{theorem}
\begin{theorem}[Basic Structure of Conditional Mean and Conditional Variance]
\hfill\\\normalfont For return $r_t$, denote conditional mean by $u_t$
\[
r_t= \mu_t + u_t
\]
and usually we model $\mu_t$ by $\ARMA(p,q)$,
\[
\mu_t = \phi_0 + \sum_{i=1}^p \phi_i r_{t-i} + \sum_{i=1}^q \psi_i u_{t-i}
\]
Volatility models are concerned with modelling of
\[
\sigma_t^2  = \cov(r_t, \mid r_{t-1}, \ldots) = \cov(u_t\mid r_{t-1}, r_{t-2},\ldots)
\]
the conditional variance of the return.\\
We can use univariate volatility models such that
\begin{itemize}
  \item ARCH
  \item GARCH
  \item GARCH-M
  \item $\cdots$
\end{itemize}
\end{theorem}
\begin{theorem}[ARCH Model]
\hfill\\\normalfont 
\begin{align*}
r_t &= \mu_t + u_t\\
u_t &=\sigma_t\epsilon_t\\
\sigma_t^2 &=\omega + \alpha_1u_{t-1}^2 +\cdots + \alpha_mu_{t-m}^2
\end{align*}
where $\{\epsilon_t\}$ is a sequence of iid random variables, with
\[
E(\epsilon_t) = 0, \cov(\epsilon_t) = 1
\]
and $\omega>0$, and $\alpha_i\geq 0$ for $i>0$.
\[
\epsilon_t \perp \{\sigma_s, s\leq t\}
\]
We call $u_t=r_t-\mu_t$ \textbf{residuals} and $\epsilon_t$ \textbf{standardized residuals}.\\
Commonly used distribution for $\epsilon_t$ can be standardized normal, or standardized student-$t$.\\
Let's take $\ARCH(1)$ as example:
\[
r_t = u_t + \mu_t,\;\; u_t=\sigma_t\epsilon_t,\;\;\sigma_t^2 = \omega + \alpha_1u_{t-1}^2
\]
where $\omega>0$and $\alpha_1\geq 0$.\\
We have
\begin{itemize}
  \item $E(u_t) = 0$
  \item $u_t$ is called residuals, $\epsilon_t$ standardized residuals.
  \item Model can be written as
  \[
u_t^2 = \omega + \alpha_1 u_{t-1}^2 + \eta_t
  \]
where $\eta_t = \sigma_t^2(\epsilon_t^2 - 1)$.
  \item If $\alpha_1<1$, $E(u_t^2)$ is constant, and $u_t$ also stationary.
  \item Given $\cov(u_t)$ is constant for all $t$, then
  \[
\cov(u_t) = \frac{\omega}{(1-\alpha_1)}\text{ if } 0 < \alpha_1 < 1
  \] 
  \item Under normality of $\epsilon_t$, and given the moments are constants for all $t$, we have
  \[
\mu_4 = \frac{3\omega^2(1+\alpha_1)}{(1-\alpha_1)(1-3\alpha_1^2)}
  \]
provided $0<\alpha_1^2<1/3$. However, the 4th moment does not exist if $\alpha_1^2>1/3$.
  \item THe kurtosis is
  \[
\frac{\mu_4}{\mu_2^2} = 3\frac{(1-\alpha_1^2)}{(1-3\alpha_1^2)}>3
  \]
  which exmplains the heavy-tailedness of data.
\end{itemize}
We can use MLE to estimate $\omega, \alpha_i, i = 1,\ldots, m$, and by AIC, BIC, we can choose a suitable $m$.\\
The prediction of $\sigma_{t+h}^2$ can be done using the recursion relationship. 
\end{theorem}
\subsection{GARCH Model}
A general $\ARMA(p,q) + \GARCH(m,s)$ model is
\[
r_t = \mu_t + u_t, u_t = \sigma_t \epsilon_t
\]
\[
\sigma^2 = \omega + \sum_{i=1}^m \alpha_i u_{t-i}^2 +\sum_{j=1}^s \beta_j\sigma_{t-j}^2 
\]
where $\mu_t$ is $\ARMA(p,q)$ modek
\[
\mu_t = \phi_0 + \sum_{i=1}^p \phi_i r_{t-i} + \sum_{i=1}^q \psi_i u_{t-i}
\]
where $\epsilon_t$ IID with $E(\epsilon_t) = 0$ and $E(\epsilon_t^2) = 1$, $\omega > 0$, $\alpha_i\geq 0$, $\beta_j\geq 0$, and
\[
\sum_{i=1}^{\max(m;s)} (\alpha_i + \beta_i) < 1
\]
Let $\eta_t = u_t^2 -\sigma_t^2$, then $\{\eta_t\}$ is uncorrelated series. GARCH model becomes
\[
u_t^2 = \omega + \sum_{i=1}^{\max(m;s)}(\alpha_i + \beta_i)u_{t-i}^2 + \eta_t -\sum_{j=1}^s \beta_j\eta_{t-j}
\]
The probabilities properties for $\GARCH(1,1)$ include
\[
\sigma^2_t =\omega + \alpha_1 u_{t-1}^2 + \beta_1 \sigma_{t-1}^2
\]
\begin{itemize}
\item Weakly stationary: $0\geq \alpha_1$, $\beta_1\leq 1$, $\alpha_1 + \beta_1 <1$
\item Volatility clustering
\item Unconditional variance 
\[
E(u_t^2) = \frac{\omega}{1-\alpha_1-\beta_1}
\]
\item Heavy tails: if $\epsilon_t \sim N(0,1)$, and $1-2\alpha_1^2 -(\alpha_1+\beta_1)^2 > 0$, then
\[
\frac{E(u_t^4)}{[E(u_t^2)]^2} = \frac{3[1-(\alpha_1 + \beta_1)^2]}{1-(\alpha_1 + \beta_1)^2 -2\alpha_1^2} > 3
\]
\end{itemize}
\subsection{GARCH-M MOdel}
Garch In-Mean(GARCH-M) Model is
\[
r_t = \mu_t + c\sigma_t + u_t
\]
\[
u_t = \sigma_t \epsilon_t
\]
\[
\sigma_t^2 = \omega + \alpha_1u_{t-1}^2 + \beta_1 \sigma_{t-1}^2
\]
where $c$ is referred to as risk premium, which is expected to be positive.
\subsection{APARCH Model}
In some financial time series, large negative returns appear to increase volativility more than positive returns of the same magnitude do.\\
$APARCH(p,q)$ model for conditional standard deviation is
\[
\sigma_t^\delta = \omega + \sum_{i=1}^p \alpha_i(|u_{t-1}| - \gamma_i u_{t-1})^\delta + \sum_{j=1}^q \beta_j\sigma_{t-j}^\delta
\]
where $\delta>0$ and $-1<\gamma_j < 1$. Note that $\delta = 2$ and $\gamma_i = 0$ gives standard GARCH model.\\
\subsection{Value At Risk Conditional on the Past}
Given past information $r_1, \ldots, r_{t-1}$, for a small $\alpha>0$, we define the value at risk as
\[
\VaR_t(\alpha) = - \max\{v: P(r_t\leq v\mid r_{t-1}, \ldots) \leq \alpha\}
\]
If we assume $r_t = \mu_t + \sigma_t \epsilon_t$, and $\epsilon_t\perp \sigma_t$, then
\[
\VaR_t(\alpha) = -\mu_t -\sigma_tz_{\alpha} = -\mu_t +\sigma_t \VaR_\alpha(\epsilon)
\]
where $z_\alpha$ is the $\alpha$-th quantile of $\epsilon_t$.\\
If $\epsilon_t\sim N(0,1)$, then
\[
z_\alpha = \Phi^{-1}(\alpha)
\]
If $t$ distribution with freedom $\nu$, then $z_\alpha t_\nu^{-1}(\alpha)/\sqrt{\frac{\nu}{\nu - 2}}$.\\
The $\VaR_{t}$ with level alpha means $P(u_t>-\VaR_t(\alpha)) = 100(1-\alpha)\%$.\\
To validate the calculation of $\VaR_t(\alpha)$, the estimated probability
\[
\hat{p} = \frac{\#\{u_t>-\VaR_t(\alpha), t = 1,\ldots, T\}}{T}
\]
should be approximately $100(1-\alpha)\%$.\\
If $\hat{p}> 100(1-\alpha)\%$, then the model is too conservative, and otherwise, model is too aggressive.\\
\subsection{Prediction of VaR}
We first predict $\sigma_{T+h}$ then, together with the distribution, calculate $\VaR$ at day $T+h$ at level $\alpha$:
\[
-\mu_{T+h\mid T} - \sigma_{T+h\mid T}Q(\alpha)
\]

\end{document}
