\PassOptionsToPackage{svgnames}{xcolor}
\documentclass[12pt]{article}



\usepackage[margin=1in]{geometry}  
\usepackage{algorithm,algpseudocode}
\newcommand{\vars}{\texttt}
\newcommand{\func}{\textrm}
\usepackage{graphicx}             
\usepackage{amsmath}              
\usepackage{amsfonts}
\usepackage{framed}               
\usepackage{amssymb}
\usepackage{array}
\usepackage{amsthm}
\usepackage{multirow}
\usepackage[nottoc]{tocbibind}
\usepackage{bm}
\usepackage{enumitem}
\usepackage{tikz}
\usepackage{pdfpages}
\usepackage{tabularx}
\usepackage{verbatim}
\algdef{SE}[SUBALG]{Indent}{EndIndent}{}{\algorithmicend\ }%
\algtext*{Indent}
\algtext*{EndIndent}
\newcolumntype{C}[1]{>{\centering\let\newline\\\arraybackslash\hspace{0pt}}m{#1}}
\newcommand\norm[1]{\left\lVert#1\right\rVert}
\setlength{\parindent}{0cm}
\setlength{\parskip}{0em}
\newcommand{\ind}{\hspace*{15pt}}
\newcommand{\Lim}[1]{\raisebox{0.5ex}{\scalebox{0.8}{$\displaystyle \lim_{#1}\;$}}}
\newtheorem{definition}{Definition}[section]
\newtheorem{theorem}{Theorem}[section]
\newtheorem{notation}{Notation}[section]
\theoremstyle{definition}
\DeclareMathOperator{\arcsec}{arcsec}
\DeclareMathOperator{\arccot}{arccot}
\DeclareMathOperator{\arccsc}{arccsc}
\DeclareMathOperator{\spn}{Span}
\DeclareMathOperator{\diff}{d}
\setcounter{tocdepth}{1}
\begin{document}
\title{Revision notes - CS3203}
\author{Ma Hongqiang}
\maketitle
\tableofcontents

\clearpage
%\twocolumn
\section{SIMPLE}
\subsection{Language Syntax}
\textsc{SIMPLE} is the target language of CS3203. \\
We use suffix $+$ to denote $1$ to many. We use suffix $*$ to denote $0$ to many.\\
We further define the following lexical tokens:
\begin{itemize}
	\item LETTER: A-Z $|$ a-z
	\item DIGIT: 0-9
	\item NAME: LETTER(LETTER | DIGIT)$*$
	\item INTEGER: DIGIT$+$
\end{itemize}
Its \textbf{grammar} is defined as below:
\begin{itemize}
	\item \textbf{program}: procedure$+$
	\item \textbf{procedure}: \texttt{procedure procedureName \{ }stmtLst\texttt{ \}}
	\item \textbf{stmtLst}: statement$+$
	\item \textbf{statement}: read $|$ print$|$ call $|$ while $|$ if $|$ assign
	\item \textbf{read, print}(I/O): \texttt{read }\textit{variableName}\texttt{;}, \texttt{print }\textit{variableName}\texttt{;}
	\item \textbf{call}(function call): \texttt{call }\textit{procedureName}\texttt{;}
	\item \textbf{while}(loop): \texttt{while ( }cond\_expr\texttt{ ) \{ }stmtLst\texttt{ \}} 
	\item \textbf{cond\_expr}: Can be one of the following forms. Note that cond\_expr requires bracketing \textit{always}.
	\begin{itemize}
		\item rel\_expr
		\item \texttt{!( }cond\_expr\texttt{ )}
		\item \texttt{( }cond\_expr\texttt{ )}
		\item \texttt{( }cond\_expr\texttt{ ) \&\& ( }cond\_expr\texttt{ )}
		\item \texttt{( }cond\_expr\texttt{ ) || ( }cond\_expr\texttt{ )}
	\end{itemize}
	\item \textbf{rel\_expr}: gt $|$ gte $|$ lt $|$ lte$|$ eq $|$ neq
	\item \textbf{gt, gte, lt, lte, eq, neq}(=op): rel\_factor op rel\_factor
	\item \textbf{rel\_factor}: \textit{variableName} $|$ \textit{constant} $|$ expr
	\item \textbf{expr}: plus $|$ minus $|$ times $|$ div $|$ mod $|$ ref. Specifically, it is of the following forms:
	\begin{itemize}
		\item expr and term
		\item expr minus term
		\item term
	\end{itemize}
	\item \textbf{term}: One of the following forms:
	\begin{itemize}
		\item term times factor
		\item term div factor
		\item term mod factor
		\item factor
	\end{itemize}
	\item \textbf{factor}: \textit{variableName} $|$ \textit{constant} $|$ \texttt{( }expr\texttt{ )}
	\item \textbf{plus, minus, times, div, mod, ref}(=op): expr op expr
	\item \textbf{ref}: \textit{variableName} $|$ \textit{constant}
	\item \textbf{if}: \texttt{if ( }cond\_expr\texttt{ ) then \{ }stmtLst\texttt{ \} else \{} stmtLst\texttt{ \}}
	\item \textbf{assign}: \textit{variableName}\texttt{ = } expr\texttt{;}
	\item \textbf{variableName}, \textbf{procedureName}: NAME
	\item \textbf{constant}: INTEGER
\end{itemize}
Note, we represent operators as such:
\begin{itemize}
	\item \textbf{and, or, not}: \texttt{\&\&}, \texttt{||}, \texttt{!}
	\item \textbf{gt, gte, lt, lte, eq, neq}: \texttt{>}, \texttt{>=}, \texttt{<}, \texttt{<=}, \texttt{==}, \texttt{!=}
	\item \textbf{plus, minus, times, div, mod}: \texttt{+}, \texttt{-}, \texttt{*}, \texttt{/}, \texttt{\%}
\end{itemize}
\subsection{Programme Knowledge Base}
\begin{definition}[Program Design Entities]
\hfill\\\normalfont The following list contains all \textbf{programme design entities}:
\begin{itemize}
	\item procedure
	\item stmtLst
	\item stmt
	\item read, print, assign, call, while, if
	\item variable
	\item constant
\end{itemize}
\end{definition}
\begin{definition}[Programme Design Abstraction]
\hfill\\\normalfont \textbf{Program design abstractions} are \textit{relationships} betwwen programme design entities.
\end{definition}
Following are 4 basic design abstractions:
\begin{definition}[Follow]
\hfill\\\normalfont For any statement $s_1, s_2$, 
\begin{itemize}
	\item \textbf{Follows}($s_1$, $s_2$) holds if they are at the same nesting level, in the same statement list, and $s_2$ appears \textit{immediately} after $s_1$.
	\item \textbf{Follows}$^*$($s_1$, $s_2$) holds \textit{if} \textbf{Follows}($s_1$, $s_2$) \textit{or} \textbf{Follows}($s_1$, $s$) and \textbf{Follows}$^*$($s$, $s_2$) for some statement $s$.
\end{itemize}
\end{definition}
\begin{definition}[Parent]
\hfill\\\normalfont Forany statement $s_1, s_2$, 
\begin{itemize} 
	\item \textbf{Parent}($s_1$, $s_2$) holds if $s_2$ is directly nested in $s_1$.
	\item \textbf{Parent}$^*$($s_1$, $s_2$) holds \textit{if} \textbf{Parent}($s_1$, $s_2$) \textit{or} \textbf{Parent}($s_1$, $s$) and \textbf{Parent}$^*$($s$, $s_2$) for some statement $s$.
\end{itemize}
\end{definition}
\begin{definition}[Use]
\hfill\\\normalfont 
\end{definition}
\end{document}